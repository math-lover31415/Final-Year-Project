%----------------------------------------------------------------------------------------
%    PACKAGES AND THEMES
%----------------------------------------------------------------------------------------

\documentclass[aspectratio=169,xcolor=dvipsnames]{beamer}
\usetheme{SimpleDarkBlue}

\usepackage{hyperref}
\usepackage{graphicx} % Allows including images
\usepackage{booktabs} % Allows the use of \toprule, \midrule and \bottomrule in tables

%----------------------------------------------------------------------------------------
%    TITLE PAGE
%----------------------------------------------------------------------------------------

\title{Real Time Underwater Image Restoration
using Co-Operational Regressor Networks}
\subtitle{CoRe-Nets}

\author{Aravind Ashokan, Harishanker S Nair, Pradyumn R Pai, Pranav P S}

\institute
{
    Department of Computer Science and Engineering \\
    Govt. Model Engineering College
}
\date{August 4th, 2025} % Date, can be changed to a custom date

%----------------------------------------------------------------------------------------
%    PRESENTATION SLIDES
%----------------------------------------------------------------------------------------

\begin{document}

\begin{frame}
    % Print the title page as the first slide
    \titlepage
\end{frame}


\section{Objective}
\begin{frame}{Objective}
\begin{itemize}
    \item Develop a GPU-accelerated application for restoring degraded underwater images.
    \item Address challenges like noise, blur, color distortion, and light scattering.
    \item Perform real-time video enhancement.
\end{itemize}
\end{frame}

\begin{frame}{Applications of the Technology}
\begin{itemize}
    \item Enhancing navigation for unmanned underwater vehicles (UUVs).
    \item Augmenting low-light and nighttime surveillance footage.
    \item Marine ecosystem studies and archaeological exploration.
    \item Inspection of subsea pipelines and offshore infrastructure.
    \item Integration with autonomous underwater vehicles (AUVs) and ROVs.
\end{itemize}
\end{frame}

\begin{frame}{Methodology}
\begin{itemize}
    \item Underwater images are decomposed into high-frequency(HF) and low-frequency(LF) components using Discrete Cosine Transform
    \item HF captures edge details and LF captures global appearance
    \item A CNN directly enhances the HF component to restore image sharpness.
    \item The LF network estimates a Joint Component Map (JCM), which integrates the transmission map and background light. This bypasses errors from estimating them separately and is used to reconstruct the LF component.
    \item The enhanced HF and LF components are then combined to form a preliminary enhanced image.
    \item The preliminary image is passed through a refinement network to improve visual quality, correcting residual distortions and ensuring better color fidelity.
\end{itemize}
\end{frame}

\begin{frame}{Potential}
\begin{itemize}
    \item Broader applications: medical imaging, low-light surveillance, atmospheric haze removal.
    \item Transformative potential for challenging imaging tasks across domains.
\end{itemize}
\end{frame}


\begin{frame}{References}
\nocite{*}
\bibliography{references}
\bibliographystyle{plain}
\end{frame}

\end{document}