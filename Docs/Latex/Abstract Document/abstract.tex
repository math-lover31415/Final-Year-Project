\documentclass[a4paper,12pt]{article}
\usepackage{amsmath}
\usepackage{geometry}
\geometry{top=1in, bottom=1in, left=1in, right=1in}
\usepackage{setspace}
\usepackage{titlesec}

\titleformat{\section}
  {\normalfont\fontsize{12}{15}\bfseries}{\thesection}{1em}{}

\title{\textbf{A WAVELET-BASED DUAL-STREAM NETWORK FOR UNDERWATER
IMAGE ENHANCEMENT}}

\begin{document}

\begin{center}
    \textbf{Model Engineering College, Ernakulam}\\
    \textbf{Department of Computer Engineering}\\
    B. Tech. Computer Science \& Engineering\\
    \textbf{CSQ413 SEMINAR}\\
    \vspace{0.5cm}
    \LARGE{\textbf{A WAVELET-BASED DUAL-STREAM NETWORK FOR UNDERWATER
IMAGE ENHANCEMENT }}\\
    \vspace{0.5cm}
    \large{MDL22CS049 CSA16 Aravind Ashokan}\\
    \large{17 July 2025}\\
    \vspace{0.5cm}
    \hrule 
    \vspace{0.5cm}
    \footnotesize{\textbf{Keywords: } Image enhancement, wavelet decomposition, dual-stream network, discrete wavelet transform (DWT), multi-color space fusion}
\end{center}

\section*{Abstract}
Underwater image enhancement is a critical area of research in computer vision because images captured underwater often suffer from severe visual degradation caused by light absorption and scattering. These problems appear as color casts, low contrast, and blurring, which limit the usability of underwater images in fields such as marine exploration, biological research, and underwater infrastructure inspection and maintenance.\\

A wavelet-based dual-stream network is designed to address the complex problems present in underwater images. The method uses discrete wavelet transform to decompose each image into multiple frequency sub-bands, effectively separating global color information from fine image details. A multi-color space fusion network operates on the low-frequency components to correct color distortions by combining representations from RGB, HSV, and Lab color spaces. At the same time, a detail enhancement network restores sharpness by processing the high-frequency sub-bands separately. The outputs of both networks are combined using inverse wavelet transform to produce the final enhanced image.\\

Compared to traditional physics-based approaches and earlier learning-based methods, this dual-stream framework shows better performance in removing color casts and recovering image clarity while keeping computational complexity low. The proposed model achieves competitive quantitative results and produces visually natural outputs without introducing artificial effects.\\

The method holds significant potential for applications that rely on accurate underwater perception, including autonomous underwater vehicles, inspection systems, and scientific imaging tasks.


\section*{References}
[1] Ziyin Ma and Changjae Oh, “A wavelet-based dual-stream network for underwater image enhancement”, 2022. Accessed: Jul. 10, 2025. [Online].\\
Available: https://ieeexplore.ieee.org/document/9747781

 \vspace{0.5cm}
 \hrule 
 \vspace{0.5cm}
 
\noindent
\\
\\
\\
\textbf{Staff in Charge:}\\
Dr. Murali Mohanan\\
Associate Professor\\
Department of Computer Engineering\\
Model Engineering College

\vspace{0.5cm}

\noindent
\\
\\
\textbf{Internal Guide:}\\
Mrs. Gency Anoop\\
Assistant Professor\\
Department of Computer Engineering\\
Model Engineering College

\end{document}