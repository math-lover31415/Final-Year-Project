\subsection{\textit{Lightweight Underwater Image Enhancement via Impulse Response of Low-pass Filter Based Attention Network\cite{Tun_2024}}}

\paragraph{Abstract:}
This paper proposes an \textbf{improved Shallow-UWnet model} for underwater image enhancement targeting resource-constrained underwater robots:
\begin{enumerate}
    \item \textbf{Skip Connection Enhancement:} Incorporates raw underwater images and impulse response of low-pass filter (LPF) to solve vanishing gradient problems.
    \item \textbf{Attention Integration:} Integrates parameter-free SimAM attention modules into each Convolution Block for enhanced visual quality.
    \item \textbf{Lightweight Design:} Achieves comparable performance with \textbf{216,000 parameters} (fewer than original Shallow-UWnet) and \textbf{0.05\,s testing time} per image.
\end{enumerate}
Demonstrates superior results with \textbf{PSNR = 27.87}, \textbf{SSIM = 0.84}, and \textbf{UIQM = 2.96} on the EUVP-Dark dataset.

\paragraph{Methodology:}
\begin{itemize}
    \item \textbf{Architecture Enhancement:}
          \begin{itemize}
              \item Modifies Shallow-UWnet by adding skip connections that concatenate raw underwater images with LPF impulse responses.
              \item Reduces ConvBlock features from 61 to 58 to accommodate additional input channels.
              \item Maintains three successive ConvBlocks with ReLU activation and dropout regularization.
          \end{itemize}

    \item \textbf{Low-Pass Filter Integration:}
          \begin{itemize}
              \item Evaluates four LPF variants: Sparsity-based LPF (SLPF), Direct LPF (DLPF), Gaussian LPF (GLPF), and Butterworth LPF (BLPF).
              \item \emph{SLPF formulation:} Power spectrum sparsity
                    \[
                        S = \frac{P_a}{P_h + P_v}, \quad \gamma = \lambda S.
                    \]
              \item \emph{Frequency response:}
                    \[
                        H_S(\omega_1,\omega_2)=
                        \begin{cases}
                            1, & P(\omega_1,\omega_2)\le\gamma, \\
                            0, & \text{otherwise}.
                        \end{cases}
                    \]
          \end{itemize}

    \item \textbf{SimAM Attention Module:}
          \begin{itemize}
              \item Parameter-free 3D attention mechanism based on energy theory.
              \item \emph{Energy calculation:}
                    \[
                        \varepsilon_T = \frac{4(\rho^2 + \alpha)}{(T - \eta)^2 + 2\rho^2 + 2\alpha},
                    \]
                    where $\eta = \tfrac{1}{N}\sum_i y_i$ and $\rho^2 = \tfrac{1}{N}\sum_i(y_i-\eta)^2$.
              \item \emph{Attention output:}
                    \[
                        \widetilde{Y} = \sigma\!\bigl(\tfrac{1}{E}\bigr)\odot Y.
                    \]
          \end{itemize}
\end{itemize}

\paragraph{Advantages:}
\begin{itemize}
    \item \textbf{Lightweight architecture:} 216,000 parameters vs.\ 219,456 in original Shallow-UWnet.
    \item \textbf{Fast processing:} 0.05\,s per image enabling real-time enhancement.
    \item \textbf{Robust performance:} Comparable or superior PSNR/SSIM/UIQM across EUVP-Dark, UFO-120, and UIEB.
    \item \textbf{Noise reduction:} Better distinguishes image content from noise.
\end{itemize}

\paragraph{Disadvantages:}
\begin{itemize}
    \item \textbf{Marginal PSNR improvement:} Only slight gain over baseline (27.87 vs.\ 27.86 on EUVP-Dark).
    \item \textbf{Color artifacts:} Overcontrast with reddish hue in heavily hazy regions persists.
    \item \textbf{Dataset dependency:} Trained mainly on EUVP, may generalize poorly to other conditions.
    \item \textbf{LPF variant selection:} No clear guidance on optimal filter choice; benefits similar across variants.
\end{itemize}
