\subsection{\textit{Blind Underwater Image Restoration using Co-Operational Regressor Networks}\cite{Devecioglu_2024}}

\paragraph{Abstract:}
This paper proposes \textbf{Co-Operational Regressor Networks (CoRe-Nets)} which combine two cooperating networks: the Apprentice Regressor (AR) for image restoration, and the Master Regressor (MR) that estimates image quality (PSNR) and provides feedback to the AR

\paragraph{Methodology:}
\begin{itemize}
    \item Apprentice Regressor(AR)
          \begin{itemize}
              \item A U-Net like structure trained to restore underwater images.
              \item Image quality is measured by PSNR (Peak Signal-to-Noise Ratio).
              \item AR is trained to minimize a loss composed of:
                    \begin{itemize}
                        \item PSNR-based feedback from MR
                        \item Actual PSNR compared to ground truth
                        \item Focal Frequency Loss (FFL) used to align patterns in the frequency domain and optimize image restoration by focusing on major spectral components
                    \end{itemize}
          \end{itemize}

    \item Medium Transmission Based Decoder
          \begin{itemize}
              \item Use a reverse medium transmission map as an attention mask
                    \begin{itemize}
                        \item Medium transmission T(x) represents how much light from the scene at pixel x reaches the camera after being attenuated and scattered in water.
                        \item It can be measured mathematically based on scene radiance and background light
                        \item Reverse Medium Transformation (RMT) is measured as RMT(x) = 1-T(x) and represents how much degradation is caused in the image due to water
                    \end{itemize}
              \item Combine encoder features with the RMT map to emphasize degraded areas
              \item Reconstruct output via resifual enhancement and upsampling layers
          \end{itemize}
\end{itemize}

\paragraph{Advantages:}
\begin{itemize}
    \item \textbf{High performance} \\
          Has 24.54 dB PSNR

    \item \textbf{Low Complexity} \\
          Has only 7.2 milliom parameters

    \item \textbf{Real Time capable} \\
           The processing time to restore a 256x256x3 image takes around 6.1 msec for a single CPU implementation

\end{itemize}

\paragraph{Disadvantages}

\begin{itemize}
    \item \textbf{No Physics-Guided Constraints} \\
          CoreNets are purely data driven and have no physics guided constraints

    \item \textbf{Feedback Loop} \\
          Cooperative setup can create interdependence and non linearity in training, making it hard to tune convergence behaviour

    \item \textbf{Dataset Diversity} \\
          The model was trained and tested on the LSUI dataset, and not on other challenging underwater datasets like UIEB, EUVP, or in-the-wild underwater conditions, limiting the paper's claims of robustness.

    \item \textbf{Lack of Real Ground Truth for Transmission Maps} \\
          Medium transmission map is estimated based on heuristics and may not capture true degradation levels
\end{itemize}